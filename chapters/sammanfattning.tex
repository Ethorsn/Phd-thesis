Portföljer med finansiella instrument och diversifiering går oftast hand i hand.
Diversifiering är ett, om inte det bästa, verktyget för att minska risken i en portfölj.
Om en av investeringarna skulle gå dåligt spelar det ingen roll för den utgör en sådan liten del av portföljen.
Modern Portföljteori (MPT) är ett ramverk för att konstruera diversifierade portföljer.
Ett hinder är dock att MPT använder okända parametrar.
Dessa okända parametrar är de två första momenten av tillgångarnas avkastningsfördelning.
När dessa ersätts av skattare introduceras estimerings-osäkerhet.
Om en investerare inte förstår osäkerheten finns det en risk att den dominerar alla resultat som portföljen, eller strategin, skulle kunna påvisa.
Det finns inget sätt att avgöra om strategin fungerar eller inte.

Denna avhandling innehåller fem artiklar.
De innehåller resultat som kan hjälpa en investerare att hantera skattningsosäkerhet i oändliga portföljer samt, i vissa fall, beskriva portföljernas fördelning för ett ändligt stickprov.
Dessa resultat gör det lättare för investerare att förstå investeringsprocessen och de empiriska resultat som kan observeras i praktiken.

Artikel 1 utforskar alla portföljer i MPT-ramverket.
I artikeln härleds alla optimala portföljers empiriska fördelning, det vill säga när skattare används istället för de sanna parametrarna.
Det inkluderar fördelningen för vikterna samt den avkastning, varians och andra mått som karateriserar dessa portföljer.
Därefter härleds den asymptotiska fördelningen för alla mått samt vikter i stora dimensioner.
Denna fördelning kan ses som ett fall då en investerare diversifierar oändligt mycket i MPT-ramverket samt har oändligt mycket data gällande de tillgångar han/hon investerar i.
En simuleringsstudie visar att den asymptotiska fördelningen utgör en god approximation av fördelningen för ändliga stickprov, givet att modelantagandet håller.

Artikel 2 fortsätter vidare på artikel 1.
I denna artikel undersöks portföljen från det kvadratiskt nytto-optimeringsproblem med en riskfri tillgång. 
En riskfri tillgång kan exempelvis vara ett räntebärande konto.
Denna portfölj är också känd som den tangerande portföljen.
I artikeln härleds den empiriska fördelningen för portföljvikterna fast med en generalisering av modellen från artikel 1.
Fördelningen för tillgångarna antas vara skev-normal.
Resultaten visar att skevhet gör att de empiriska vikterna i portföljen är i genomsnitt fel.
Detta sker i genomsnitt för ändliga stickprov men den empiriska portföljen skattar rätt objekt i stora dimensioner. 

Artikel 3 ger en lösning på problemet att äga en portfölj för att sedan gå över till en annan.
I artikeln utvecklas en metod som omfördelar portföljen på givna tidpunkter.
Den gör detta genom att minimera den framtida variansen för Minsta Varians (GMV) portföljen, givet att en investeraren redan äger en portfölj.
Portföljen som ägs i denna stund kan vara deterministisk eller en skattad GMV-portfölj.
En omfattande simuleringsstudie visar att denna metod kan uppnå goda resultat i termer av att skatta den relativa förlusten.
Den relativa förlusten är en enkel utveckling av portföljvariansen.
Metoden utvärderas därefter med hjälp av marknadsdata.
Den lyckas bäst i att ge minsta framtida varians samt ge de minsta förändringarna i portföljvikterna bland de olika metoderna i jämförelsegruppen.
Metoden har implementerats i ett R-paket som heter DOSPortfolio och finns tillgängligt på CRAN. 

Artikel 4 härleder olika egenskaper för två olika prestandamått av tre olika skattare för GMV-portföljer.
Prestandamåtten är den framtida variansen samt den framtida relativa förlusten för GMV-portföljerna.
Det tidigare nämnda prestandamåttet används närpå alltid för att utvärdera GMV-portföljer med empirisk data.
Resultaten visar att den relativa förlusten inte behöver lika strikta antaganden för att konvergera i stora dimensioner.
Detta mått kan därför täcka flera modeller till skillnad från variansen.
Prestandamåtten används sedan för att bestämma en ordning på de olika portföljerna.
Dessa resultat verifieras i en simuleringsstudie samt med empirisk data. 

Artikel 5 utvidgar en av GMV-portföljerna från artiklarna 3 och 4.
I den här artikeln introduseras en Thikonov-regularisering på portföljvikterna vilket resulterar i en Ridge-liknande skattare för den empiriska kovariansmatrisen.
Detta kombineras sedan med den linjära shrinkage-metoden från artiklarna 3 och 4.
Denna portfölj undersöks sedan i en omfattande simuleringsstudie och dess prestanda studeras med empirisk data.
Simuleringsstudien visar att metoden är jämförbar med ett flertal metoder.
Den empiriska studien visar att metoden som utvecklas ger lägre skattningar av framtida varians än ett flertal referensportföljer samt att den visar bra prestanda gällande portföljvikterna.