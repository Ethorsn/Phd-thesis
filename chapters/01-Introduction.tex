% General introduction to why portfolio theory and allocations are hard
% ca 1page
Splitting a budget between your possible or existing investments is always a decision that needs to be made.
How should you spend your cash?
This thesis is about how to make allocations when you have many possible choices and do not know how their payoffs will behave in the future.
That turns out to be very complicated due to the fact that you have so many unknowns.

A portfolio is a combination of assets.
An asset might be a house, a stock, the value of a contract or any object with a price that you could potentially sell.
Lets take the first asset as an example. 
The portfolio consists of one asset, a house in Stockholm, Sweden.
It has been a great investment\footnote{As Metallica says, "Sad but true" \url{https://www.maklarstatistik.se/omrade/riket/stockholms-lan/\#/villor/48m-prisutveckling}}.
However, given that we are close to all-time highs its uncertain how it will be valued in the future.
Will its price continue to increase or will there be a decrease, how much will that change be?
There are many unknowns.
Lets assume that the choice is between keeping the cash or spend it on the house.
To make a quantitative assesment of the investment we need to specify how likely a increase is and what value it has after the increase.
For a house that might be feasible.
There are many houses that have been renovated, sold and look similar to ours.
If we take the second asset we listed, a stock, that might not be the case.
There is only one stock with one accompanying time series.
We can only evaluate the investment based on the companies history and future prospects.
This makes the analysis harder since there are no ''identical'' copies of our stock that we can look at and extract insight from.
We can only observe the price as it develops and try to make educated guesses to where it will be tomorrow on within a year.
Making a good guess is hard.
Markets are very hard to predict.

In this thesis we do not predict tomorrows price but try to answer what portfolio of assets you should hold in the long run.
This thesis builds upon the foundation of \citet{markowitz1959portfolio} Modern Portfolio Theory (MPT).
In his seminal work he argued that any portfolio which simply maximizes its profit will result in a naive solution.
Investing all your capital in the asset with the highest return is not sensible if you do not know the future.
Such an investment will cause you to take an extreme amount of risk. 
Markowitz argued that any well diversified portfolio should be preferred to any non diversified portfolio. 
In essence, a diversified portfolio contains many assets that are hopefully not correlated.
Such a portfolio can be obtained through many different procedures.
If the future is unknown then one of the best model(s) we have for it is stochastic.
He proposed the use of the first two moments for the allocation problem.
If an asset has high return on average, it might make sense to invest a lot in it although not at the cost of large amounts of risk. 
If an asset is not risky, then it would always make sense to invest in it.


To practically use a portfolio from the MPT framework we need to estimate the parameters of the model.
Since MPT use the first two moments we need to estimate the mean vector and the covariance matrix.
However, using these instead of the true parameters of the model introduces estimation uncertainty into portfolio allocation problem.
If you have a small dataset it is unlikely that you will hold the correct portfolio.
\citet{okhrin2006distributional} derived the exact sampling distribution of the global minimum variance (GMV) portfolio, a very important portfolio in the MPT framework.
It is the portfolio which provides the smallest possible variance in the MPT context.
That portfolio has been extensively researched in different contexts.
\citet{kempf2006estimating} show that an estimate of the GMV portfolio can be obtained through the classical plug-in approach with sample moments or by a regression approach.
\citet{frahm2010} constructs an estimator which combines the GMV portfolio a target portfolio. 
They use properties of the Wishart distribution and develop a regularization method for the GMV portfolio, e.g. how this combination between the target and the GMV can be done optimally.

Diversification is one, if not the best, risk management tool there is.
High-dimensional portfolios can be thought of as a consequence of diversification.
It should, in theory, decrease the risk of the portfolio. 
That is not always the case.
By introducing one new asset to our portfolio we need to estimate more parameters than the number of assets in the portfolio. 
The covariance matrix suffers from the curse of dimensionality. 
In terms of estimation uncertainty, this does not scale well.
\citet{lw17} and \citet{bodnar2018estimation} consider estimating very large, or high-dimensional, GMV portfolios using two different approaches.
The former assumes that the eigenvectors are known and consider a nonlinear rotational-invariant estimation method for the covariance matrix. 
The latter develops a regularization method much like \citet{frahm2010} but does so for high-dimensional portfolios.
\citet{ding2021high} use a factor model to estimate the GMV portfolio and describe the implications of the factor structure in the high-dimensional setting.
The factor model is common in finance because of its use in arbitrage pricing theory (see e.g. \citet{ross2013arbitrage}).
\citet{golosnoy2019exponential} and \citet{cai2020high} use data on a very high frequency to estimate a sequence of sample covariance matrices and use these for high-dimensional GMV portfolios. 
The high-dimensional GMV portfolio appears in other fields as well, such as signal processing where it goes under the name beamformer (see e.g. \citet{LiStoicaWang2004}). 

The GMV portfolio is a good portfolio since it does not depend on the mean vector.
Estimating the mean is hard (see e.g. \citet{merton1980estimating}, \citet{best1991sensitivity}).
However, it is only one out of many portfolios in the MPT framework.
\citet{el2010high} investigated the weights of the MPT framework when linear constraints are introduced to the portfolio allocation problem.
That can include position restrictions as well as a investors desire for a given return.
\citet{bodnar2020sampling} investigates the distribution of all portfolios in the MPT framework.
They do so in both a high-dimensional and finite sample setting.
\citet{bodnarokhrinparolya2020} considers high-dimensional portfolio but use another portfolio allocation problem.
The setting is similar to \citet{bodnar2018estimation} where they regularize the quadratic utility portfolio in the high-dimensional setting towards a target portfolio.

This thesis is structured as follows. 
In Chapter \ref{ch:MPT} we present the framework we will work within, namely MPT. 
Originally MPT use information which are not available. 
It relies on parameters which we do not know. 
The subsequent chapter, Chapter \ref{ch:estim} displays what happens when we replace these parameters with estimates.
It does so under some specific models that we consider in this thesis.
Chapter \ref{ch:highdim} presents what happens in the MPT framework when we try to diversify as much as possible.
We work with infinite dimensional portfolios.
The last two chapters, Chapter \ref{ch:papersummary} and \ref{ch:future} are summaries of the papers presented in this thesis as well as a summary of future possible research.
All code for this thesis is available on \href{https://github.com/Ethorsn/Phd-thesis}{Github, here.}
