% General introduction to why portfolio theory and allocations are hard
% ca 1page
Splitting a budget between your possible or existing investments is always a decision that needs to be made.
How should you spend your cash?
This thesis is about how to make these decisions.
It is about how to make them when you have many possible choices and do not know how their payoffs will behave in the future.
That turns out to be very complicated simply because we have many choices and we do not know their future payoffs and its behaviour.
As an example, we start with a single investment case. 
Your imaginary house in Sweden has increased drastically in value\footnote{As Metallica says, "Sad but true" \url{https://www.maklarstatistik.se/omrade/riket/stockholms-lan/\#/villor/48m-prisutveckling}}.
How will it be valued in the future?
Will it continue to increase or will there be a decrease/crash in prices?
If there is an increase or a decrease, how much will it be?
There are so many unknowns, which is exactly the issue.
We need to describe how an asset, in this case the house, will behave in the future.
The future is uncertain and we seldom know exactly what will happen.
We look back and let history give insight to how the future will behave.
Depending on how good you are as a historian, modeler or gambler you can usually get something out of history and the data it brings.
It will never be perfect but lets assume that you are right 51\% of the time.

Lets fast forward in life.
Given your great guesses you have been lucky enough to now own a house, a country house and a allotment.
You cant be as good of a guesser on all of those assets.
Your budget needs to cover all of them.
How should you spend your cash on these assets?
You can invest in any of the three different assets and through that investment they might be worth more in the future.
Once more, you do not know that but can perhaps make an educated guess.
Given your educated guesses, should you spend all your money on one single asset?
You were right 51\% on the guesses regarding your house.
Should you continue to place all your bets on the same outcome? 
If you are uncertain in your guess then your investment might be fairly risky.
You just dont know it yet.
There are two options, spend everything on one asset or spend a little on everything.
The first option can give you the potential of a large return but also a lot of risk.
The second option can potentially give you less risk.
In this thesis we consider the second option and how you can do it when you dont know the future.

In reality making a good informed guess is hard.
Its very seldom that you are right 51\% of the time.
Markets are very hard to predict.
\textbf{Why is this important to finance and what problems can it be used to solve.}

This thesis is structured as follows. 
In Chapter \ref{ch:MPT} we present the framework we will work within, namely \citet{markowitz1959portfolio} Modern Portfolio Theory (MPT). 
Originally MPT use information which are not available. 
It relies on parameters which we do not know. 
The subsequent chapter, Chapter \ref{ch:estim} tries to display what happens when we replace these parameters with estimates.
It does so under some specific models that we consider in this thesis.
Chapter \ref{ch:highdim} presents what happens in the MPT framework when we try to diversify as much as possible.
We work with infinite dimensional portfolios.
The last two chapters, Chapter \ref{ch:papersummary} and \ref{ch:future} are summaries of the papers presented in this thesis as well as a summary of future possible research.
All code for this thesis is available on \href{https://github.com/Ethorsn/Phd-thesis}{Github, here.}
