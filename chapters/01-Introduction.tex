% General introduction to why portfolio theory and allocations are hard
% ca 1page
A portfolio is any combination of assets.
An asset might be a house, a stock, the value of a contract or any object with a price that can potentially be bought or sold.
The price of an asset develops over time.
The house prices in Stockholm, Sweden, have increased a lot in the past years\footnote{\url{https://www.maklarstatistik.se/omrade/riket/stockholms-lan/\#/villor/48m-prisutveckling}}.
However, the future price for a house is uncertain given that house prices are close to all-time highs.
Will the price continue to increase or will there be a decrease?
If there is a change, how big will that be?
Depending on the answer, the optimal choice might be to sell the house today and rent something instead.
However, to make a decision on whether or not the house should be sold today, there is a need to specify its future value. 
The future is unknown or random, in some sense.
There are many possible changes in the assets value, both in direction and magnitude.
In the context of a model there is a possibility to estimate the future price through the use of historical prices.
The decision process now includes even more uncertainty.
There is uncertainty from the choice of model, the perception of past and future prices, and the uncertainty from using estimates instead of the ''true'' parameters for the model.
If the house is one asset among many then there can be as many sources of uncertainty as there are assets, if not more.
The allocation itself, in this case selling the house, is uncertain.
Uncertainty is everywhere in decisions.

The market price of a house is rarely observed.
As a matter of fact, it is only observed when it is sold.
Even though prices are rarely observed there might be houses that are similar.
There are many different observations from the same population. 
That might not be the case for the second asset that is listed above, a stock.
The frequency of which the price of a stock is observed can be much higher.
On the other hand, there is only one stock with one accompanying timeseries.
This makes the analysis slightly different.
There can be a lot of historical data, but no identical copies of the stock that can be used to compare and extract information from.
Prices can only be observed as they develop.

In this thesis, the aim is not to predict tomorrows price of an asset but to answer what amount of a budget should be spend on the different assets in the long run.
The methods in this thesis build upon the seminal work of \citet{markowitz1959portfolio}, what is known as Modern Portfolio Theory (MPT).
Markowitz argued that any portfolio which simply maximizes its profit will result in a naive solution.
Investing all the capital in the asset with the highest return is not sensible if the future is not known.
Such an investment will be extremely risky. 
As a consequence, Markowitz argued that any well diversified portfolio should be preferred to any non-diversified portfolio. 
A well-diversified portfolio can be obtained through many different procedures.
Markowitz proposed the use of the mean and variance of the asset return distribution for the allocation problem.
If an asset has high return on average, it would make sense to invest in it if return is all that matters.
If an asset has a large variance then it is risky to invest in this asset. 
A high return might be at the cost of a large amount of risk.
If an asset is not risky, then it would always make sense to invest in it.

%%% Say something more?
%% Literature review
\subsection{Literature review}
%Sampling to GMV
To practically use a portfolio from the MPT framework the parameters of the model, the mean vector and the covariance matrix, need to be estimated.
As previously said, this introduces estimation uncertainty.
If the amount of data is small, then there is a large risk that the estimated portfolio will not be accurate and volatile as new data is observed.
The sampling distribution of the portfolio will have a large variance.
One important portfolio in the MPT framework is the Global Minimum Variance (GMV) portfolio.
This portfolio provides the smallest variance in the MPT framework.
\citet{okhrin2006distributional} derived the exact sampling distribution of the GMV portfolio. 
In the paper, the authors showed that the portfolio can have very fat tails depending on the sample size and the number of assets in the portfolio.
This is the first sign of a tradeof between diversification and the increasing influence of estimation uncertainty.
The portfolio has been extensively researched in many different contexts to cope with this feature.
\citet{frahm2010} extended the portfolio weights and constructed an estimator which combines the GMV portfolio with a target portfolio, much like \citet{stein1956} derived a regularization method for the sample mean vector. 
\citet{frahm2010} used properties of the sample covariance matrix to develop a regularization method for the GMV portfolio.
The method provides a way to combine the target and the GMV portfolio.
Furthermore, \citet{kempf2006estimating} showed that an estimate of the GMV portfolio can also be obtained through a regression approach.
Due to the vast literature in regression models, this approach has also spurred much research of the MPT framework (see, e.g., \citet{maillet2015global} and the references therein).

% high-dimensional
Diversification is a natural concept in MPT.
It was one of the motivations for it.
As more and more diversification is used, the portfolio will grow and in the most extreme case, the portfolio will contain infinitely many assets.
High-dimensional portfolios, or portfolios that contain infinite number of assets, can be thought of as a consequence of a very large amount of diversification.
Diversification is one, if not the best, risk management tool.
It should, in theory, decrease the risk of the portfolio. 
That is not always the case.
By introducing one new asset to the portfolio, more parameters than the number of assets in the portfolio need to be estimated. 
The covariance matrix suffers from the curse of dimensionality. 
In terms of estimation uncertainty, this does not scale well.
There are many ways to solve this.
\citet{lw20} and \citet{bodnar2018estimation} consider estimating high-dimensional GMV portfolios using two different approaches.
The former assumes that the eigenvectors are known and consider a nonlinear (rotational-invariant) estimation method for the covariance matrix. 
The latter develops a regularization method much like \citet{frahm2010} but does so for high-dimensional portfolios.
The mentioned papers do not assume any specific structure on the asset return distribution. 
Another common model for asset returns, which impose some structure, is the factor model (see, e.g., \citet{ross2013arbitrage}). 
A factor can be many things, such as the area in which the house is located.
\citet{ding2020high} use such a model to describe the implications of estimation to the GMV portfolio.
%\citet{golosnoy2019exponential} and \citet{cai2020high} take another approach.
%The authors of these two papers use data on a very high frequency to estimate a sequence of sample covariance matrices and use that sequence to construct high-dimensional GMV portfolios. 
The high-dimensional GMV portfolio appears in other fields as well, such as signal processing where it goes under the name beamformer (see, e.g., \citet{LiStoicaWang2004}). 

% more general MV portfolios and why GMV is most often chosen.
The largest argument for using the GMV portfolio in the MPT framework is that it does not depend on the mean vector.
Estimating the mean is hard (see, e.g., \citet{merton1980estimating}, \citet{best1991sensitivity}).
However, it is only one out of many portfolios in the MPT framework.
Including a target for the portfolio mean results in the mean-variance portfolio.
\citet{el2010high} investigated the weights of the mean-variance portfolio when linear constraints are introduced to the portfolio allocation problem.
That can include position restrictions as well as an investors desire for a given return.
\citet{bodnarokhrinparolya2020} considered high-dimensional portfolios but use another portfolio allocation problem, namely the quadratic utility.
The setting is similar to \citet{bodnar2018estimation} where the authors regularize the quadratic utility portfolio towards a target portfolio in the high-dimensional setting.
\citet{karlsson2021statistical} used an extension to the quadratic utility, where a risk-free asset is introduced.
There they derived different statistical properties of its solution, known as the tangency portfolio.

The rest of this thesis is structured as follows. 
Chapter \ref{ch:MPT} presents the framework that this thesis will work within, namely MPT. 
As previously described, MPT use information which is not available. 
It relies on parameters which are not known. 
In the subsequent chapter, Chapter \ref{ch:estim}, the models used in this thesis are introducd. 
It also introduces the more formal concept of estimation uncertainty and what it means for MPT.
Chapter \ref{ch:highdim} presents what happens in the MPT framework when the investor diviersifies as much as possible.
The portfolio will contain many, if not an infinite number, of assets.
The last two chapters, Chapter \ref{ch:papersummary} and \ref{ch:future} provide a summary of the papers presented in this thesis as well as a summary of future possible research.

All code for this thesis is available on \href{https://github.com/Ethorsn/Phd-thesis}{Github at https://github.com/Ethorsn/Phd-thesis.}
